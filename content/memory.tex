\rSec0[memory]{Memory Managment}

\indextext{diagnostic message|see{message, diagnostic}}%
\indexdefn{conditionally-supported behavior!see{behavior, conditionally-supported}}%
\indextext{dynamic type|see{type, dynamic}}%
\indextext{static type|see{type, static}}%
\indextext{ill-formed program|see{program, ill-formed}}%
\indextext{well-formed program|see{program, well-formed}}%
\indextext{implementation-defined behavior|see{behavior, implemen\-tation-defined}}%
\indextext{undefined behavior|see{behavior, undefined}}%
\indextext{unspecified behavior|see{behavior, unspecified}}%
\indextext{implementation limits|see{limits, implementation}}%
\indextext{locale-specific behavior|see{behavior, locale-specific}}%
\indextext{multibyte character|see{character, multibyte}}%
\indextext{object|seealso{object model}}%
\indextext{subobject|seealso{object model}}%
\indextext{derived class!most|see{most derived class}}%
\indextext{derived object!most|see{most derived object}}%
\indextext{program execution!as-if rule|see{as-if~rule}}%
\indextext{observable behavior|see{behavior, observable}}%
\indextext{precedence of operator|see{operator, precedence of}}%
\indextext{order of evaluation in expression|see{expression, order of evaluation of}}%
\indextext{atomic operations|see{operation, atomic}}%
\indextext{multiple threads|see{threads, multiple}}%
\indextext{normative references|see{references, normative}}

\rSec1[memory.overview]{Overview}

Valpine mostly preserves the semantics of C++ memory managment and adds new constructs for cleaner memory ownership. The main difference is references have been eliminated to simplify how addresses are managed.


\rSec1[memory.pointers]{Pointers}

Since the langauge does not use references, pointers must be used in situations where references normally would be used in C++. It is possible to explicitly take the address of something with the \& operator, but pointer assignments will implicitly take addresses as needed.

\enterexample
\begin{codeblock}
int a = 10;

//b and c both point to a
int *b = a;
int *c = &a;

//d and e both point to b
int **d = b;
int **e = &b;
\end{codeblock}
\exitexample

The reason for implicitly taking the address is to allow overloaded operators and other functions to work seamlessly with non-pointers since references do not exist.

\enterexample
\begin{codeblock}
bool operator==(const T *lhs, const T *rhs) { return lhs.value == rhs.value; }
void foo(T *t) { print(t.toString()); }

T t1;
T t2;

//without implicit address taking, this would be required
if (&t1 == &t2) print("Equal");	
foo(&t1);

//with implicit address taking, this is also legal
if (t1 == t2) print("Equal");
foo(t1);
\end{codeblock}
\exitexample

Note that there is no implicit dereferencing, and all built in operators work on primitives by value.

\enterexample
\begin{codeblock}
int *!a = new int(10);	//Scoped pointer \ref{memory.pointers.scoped}
int b = a;				//illegal, incompatible types
int c = *a;				//OK

float x = 5;
float y = 7;

//this will be comparing the the addresses of x and y since operators for
//primitive types pass by value. So this is a comparison of integer types.
if (&x > &y) print("greater");
\end{codeblock}
\exitexample

Managed pointer types are built-in and are interoperable with raw pointers.

\enterexample
\begin{codeblock}
T * p;			//raw pointer to a T
T * unsafe p;		//unsafe pointer to a T

T * ! p;		//scoped pointer to a T
T * scoped p;		//scoped pointer to a T

T * $ p;		//shared pointer to a T
T * shared p;		//shared pointer to a T

T * weak p;		//weak pointer to a T
\end{codeblock}
\exitexample

Scoped pointers can be implicitly casted to raw. All other managed pointers must be explicitly casted to a raw pointer. 

\enterexample
\begin{codeblock}
T *! p0 = new T;
T *$ p1 = new T;
T * weak p2 = p1;

T * p4 = p0;
p4 = static_cast<T*>(p1);
p4 = static_cast<T*>(p2);
\end{codeblock}
\exitexample

\Sec2[memory.pointers.raw]{Raw Pointer}

Raw pointers are simmilar to C++ raw pointers except for a few restrictions:

	1) Cannot be assigned a new expression.
	
	2) Cannot be stored except by other raw pointers.
	
	3) Cannot be deleted.
 
\enterexample
\begin{codeblock}
int * p = anotherPointer1;
p = anotherPointer2;
p = new int(10);	//illegal
\end{codeblock}
\exitexample


\Sec2[memory.pointers.unsafe]{Unsafe Pointer}

Unsafe pointers behave just like C++ raw pointers. When Valpine code interacts with C++ raw pointers, they are treated as unsafe pointers in Valpine. This may be confusing since C++ raw pointers look (and are named) the same as Valpine raw pointers, but they are not the same. Other pointer types may be casted to unsafe pointers using \tcode{static_cast<T* unsafe>(mySharedPointer)} for example.

	
\Sec2[memory.pointers.scoped]{Scoped Pointer}

Scoped pointers behave exactly like raw pointers, but automatically call the appropriate delete when exiting the scope they were created in. They can only be assigned \tcode{nullptr} or a \tcode{newexpr} expression. They are used to express single ownership. A shorthand for the \tcode{scoped} keyword is \tcode{!}.

\begin{codeblock}

T *pRaw;
{
	T *! pScoped = new T;
	pRaw = pScoped;
} //p1's memory is automatically deleted

//pRaw is now a dangling pointer

\end{codeblock}

\Sec2[memory.pointers.unique]{Unique Pointer}
Unique pointers cannot be assigned to other pointers. They automatically call the appropriate delete when exiting the scope they were created in. They can only be assigned \tcode{nullptr} or a \tcode{newexpr} expression. They are used to express single ownership. A shorthand for the \tcode{unique} keyword is \tcode{\^}.

%%TODO there is a lot not covered here such as moving ownership.

\begin{codeblock}
{
	T *^ p1 = new T;
	T *^ p2 = p1; //illegal
	T * p2 = p3; //illegal
} //p1's memory is automatically deleted

\end{codeblock}

\Sec2[memory.pointers.shared]{Shared Pointer}
Shared pointers behave similar to scoped pointers, but the automatic deletion is dependent upon a shared reference count. They can only be assigned \tcode{nullptr}, a \tcode{newexpr} expression, another shared pointer, or a weak pointer. The reference count is equal to the number of shared (strong) references to a single memory location. The memory is automatically deleted when the reference count is zero.

\begin{codeblock}

{
	T * shared p1 = new T;	//ref count = 1
	T * shared p2 = p1;		//ref count = 2
	{
		T * shared p3 = p2;	//ref count = 3
	} //ref count = 2

	T * shared p4 = p1;	//ref count = 3
} //ref count = 0, deleted

\end{codeblock}

\Sec2[memory.pointers.weak]{Weak Pointer}

Weak pointers are used for non-owning users. They can only be assigned from a shared pointer and do not increment the reference count. When the reference count from the shared pointer reaches zero and the memory is deleted, the weak pointer is set to null. When dereferencing a weak pointer, the operation should be surrounded with lock block. This atomically increments the shared reference count by one. At the end of the block, the reference count is automatically decremented. This ensures the memory location will always be dereferencable during the block. If the pointer is null during locking, the block will not execute. It is possible for the reference count to reach zero if the strong references go out of scope on other threads for example. It is illegal to dereference an unlocked weak pointer.

\begin{codeblock}

//thread 1
T * shared ps = new T;


//thread 2
T * weak pw = ps;
lock (pw)	
{
	//the reference count is guarunteed to be greater than zero throughout this block
	//even if ps goes out of scope on thread 1
	pw->doSomething();
	
} //reference count is decremented and memory is deleted if at zero


\end{codeblock}